\subsection{Extending Everything Else}\label{extending-everything-else}

\label{x-28DREF-EXT-3A-40EXTENDING-EVERYTHING-ELSE-20MGL-PAX-3ASECTION-29}

\subsubsection*{\normalfont\textcolor[HTML]{606060}{[in package DREF]}}
-
\paxlocativetypewithsource{https://github.com/melisgl/mgl-pax/blob/master/dref/src/base/extension-api.lisp\#L901}{generic-function}\paxname{resolve*}\phantomsection\label{x-28DREF-EXT-3ARESOLVE-2A-20GENERIC-FUNCTION-29}
\emph{dref}

\begin{verbatim}
Return the object defined by the definition `dref`
refers to. Signal a `resolve-error` condition by calling the
[`resolve-error`][7825] function if the lookup fails.

To keep `resolve` a partial inverse of `locate`, `define-locator` may be
necessary for `resolve`able definitions. This function is for
extending `resolve`. Do not call it directly.

It is an error for methods of this generic function to return an
`xref`.
\end{verbatim}

\begin{itemize}
\item
  \paxlocativetypewithsource{https://github.com/melisgl/mgl-pax/blob/master/dref/src/base/extension-api.lisp\#L920}{function}\paxname{resolve-error}\phantomsection\label{x-28DREF-EXT-3ARESOLVE-ERROR-20FUNCTION-29}
  \emph{\&rest format-and-args}

  Call this function to signal a \texttt{resolve-error} condition from
  the
  \href{http://www.lispworks.com/documentation/HyperSpec/Body/26_glo_d.htm\#dynamic_extent}{dynamic
  extent} of a
  \paxlink{x-28DREF-EXT-3ARESOLVE-2A-20GENERIC-FUNCTION-29}{\texttt{resolve*}}
  method. It is an error to call \texttt{resolve-error} elsewhere.

  \texttt{format-and-args}, if non-\texttt{nil}, is a format string and
  arguments suitable for
  \href{http://www.lispworks.com/documentation/HyperSpec/Body/f_format.htm}{\texttt{format}}.
\item
  \paxlocativetypewithsource{https://github.com/melisgl/mgl-pax/blob/master/dref/src/base/extension-api.lisp\#L933}{generic-function}\paxname{map-definitions-of-name}\phantomsection\label{x-28DREF-EXT-3AMAP-DEFINITIONS-OF-NAME-20GENERIC-FUNCTION-29}
  \emph{fn name locative-type}

  Call \texttt{fn} with \texttt{dref}s which can be \texttt{locate}d
  with an \texttt{xref} with \texttt{name}, \texttt{locative-type} and
  some \texttt{locative-args}. The strange wording here is because there
  may be multiple ways (and thus \texttt{xref}s) that refer to the same
  definition.

  For most locative types, there is at most one such definition, but for
  \href{http://www.lispworks.com/documentation/HyperSpec/Body/t_method.htm}{\texttt{method}},
  for example, there may be many. The default method simply does
  \texttt{(dref\ name\ locative-type\ nil)} and calls \texttt{fn} with
  result if \texttt{dref} succeeds.

  \texttt{fn} must not be called with the same (under \texttt{xref=})
  definition multiple times.

  This function is for extending \texttt{definitions} and
  \texttt{dref-apropos}. Do not call it directly.
\item
  \paxlocativetypewithsource{https://github.com/melisgl/mgl-pax/blob/master/dref/src/base/extension-api.lisp\#L957}{generic-function}\paxname{map-definitions-of-type}\phantomsection\label{x-28DREF-EXT-3AMAP-DEFINITIONS-OF-TYPE-20GENERIC-FUNCTION-29}
  \emph{fn locative-type}

  Call \texttt{fn} with \texttt{dref}s which can be \texttt{locate}d
  with an \texttt{xref} with \texttt{locative-type} with some
  \texttt{name} and \texttt{locative-args}.

  The default method forms \texttt{xref}s by combining each interned
  symbol as names with \texttt{locative-type} and no
  \texttt{locative-args} and calls \texttt{fn} if it \texttt{locate}s a
  definition.

  \texttt{fn} may be called with \texttt{dref}s that are \texttt{xref=}
  but differ in the \texttt{xref} in their \texttt{dref-origin}.

  This function is for extending \texttt{dref-apropos}. Do not call it
  directly.
\item
  \paxlocativetypewithsource{https://github.com/melisgl/mgl-pax/blob/master/dref/src/base/extension-api.lisp\#L982}{generic-function}\paxname{arglist*}\phantomsection\label{x-28DREF-EXT-3AARGLIST-2A-20GENERIC-FUNCTION-29}
  \emph{object}

  To extend \texttt{arglist}, specialize \texttt{object} on a normal
  Lisp type or on a subclass of \texttt{dref}.

  \texttt{arglist} first calls \texttt{arglist*} with its
  \texttt{object} argument. If that doesn\textquotesingle t work (i.e.
  the second value returned is \texttt{nil}), then it calls
  \texttt{arglist*} with \texttt{object} either \texttt{resolve}d (if
  it\textquotesingle s a \texttt{dref}) or \texttt{locate}d (if
  it\textquotesingle s not a \texttt{dref}).

  \begin{itemize}
  \item
    The default method returns \texttt{nil}, \texttt{nil}.
  \item
    There is also a method specialized on \texttt{dref}s, that looks up
    the
    \paxlink{x-28DREF-EXT-3ADEFINITION-PROPERTY-20FUNCTION-29}{\texttt{definition-property}}
    called \texttt{arglist} and returns its value with
    \href{http://www.lispworks.com/documentation/HyperSpec/Body/f_vals_l.htm}{\texttt{values-list}}.
    Thus, an arglist and its kind can be specified with something like

\begin{Shaded}
\begin{Highlighting}[]
\NormalTok{(}\KeywordTok{setf}\NormalTok{ (definition{-}property xref }\DataTypeTok{\textquotesingle{}arglist}\NormalTok{)}
\NormalTok{      (}\KeywordTok{list}\NormalTok{ arglist :destructuring))}
\end{Highlighting}
\end{Shaded}
  \end{itemize}

  This function is for extension only. Do not call it directly.
\item
  \paxlocativetypewithsource{https://github.com/melisgl/mgl-pax/blob/master/dref/src/base/extension-api.lisp\#L1010}{generic-function}\paxname{docstring*}\phantomsection\label{x-28DREF-EXT-3ADOCSTRING-2A-20GENERIC-FUNCTION-29}
  \emph{object}

  To extend \texttt{docstring}, specialize \texttt{object} on a normal
  Lisp type or on a subclass of \texttt{dref}.

  \texttt{docstring} first calls \texttt{docstring*} with its
  \texttt{object} argument. If that doesn\textquotesingle t work (i.e.
  \texttt{nil} is returned), then it calls \texttt{docstring*} with
  \texttt{object} either \texttt{resolve}d (if it\textquotesingle s a
  \texttt{dref}) or \texttt{locate}d (if it\textquotesingle s not a
  \texttt{dref}).

  \begin{itemize}
  \item
    The default method returns \texttt{nil}.
  \item
    There is also a method specialized on \texttt{dref}s, that looks up
    the
    \paxlink{x-28DREF-EXT-3ADEFINITION-PROPERTY-20FUNCTION-29}{\texttt{definition-property}}
    called \texttt{docstring} and returns its value with
    \href{http://www.lispworks.com/documentation/HyperSpec/Body/f_vals_l.htm}{\texttt{values-list}}.
    Thus, a docstring and a package can be specified with something like

\begin{Shaded}
\begin{Highlighting}[]
\NormalTok{(}\KeywordTok{setf}\NormalTok{ (definition{-}property xref }\DataTypeTok{\textquotesingle{}docstring}\NormalTok{)}
\NormalTok{      (}\KeywordTok{list}\NormalTok{ docstring }\VariableTok{*package*}\NormalTok{))}
\end{Highlighting}
\end{Shaded}
  \end{itemize}

  This function is for extension only. Do not call it directly.
\item
  \paxlocativetypewithsource{https://github.com/melisgl/mgl-pax/blob/master/dref/src/base/extension-api.lisp\#L1038}{generic-function}\paxname{source-location*}\phantomsection\label{x-28DREF-EXT-3ASOURCE-LOCATION-2A-20GENERIC-FUNCTION-29}
  \emph{object}

  To extend \texttt{source-location}, specialize \texttt{object} on a
  normal Lisp type or on a subclass of \texttt{dref}.

  \texttt{source-location} first calls \texttt{source-location*} with
  its \texttt{object} argument. If that doesn\textquotesingle t work
  (i.e. \texttt{nil} or
  \texttt{(:error\ \textless{}message\textgreater{})} is returned), then
  it calls \texttt{source-location*} with \texttt{object} either
  \texttt{resolve}d (if it\textquotesingle s a \texttt{dref}) or
  \texttt{locate}d (if it\textquotesingle s not a \texttt{dref}).

  \texttt{source-location} returns the last of the
  \texttt{(:error\ \textless{}message\textgreater{})}s encountered or a
  generic error message if only \texttt{nil}s were returned.

  \begin{itemize}
  \item
    The default method returns \texttt{nil}.
  \item
    There is also a method specialized on \texttt{dref}s, that looks up
    the
    \paxlink{x-28DREF-EXT-3ADEFINITION-PROPERTY-20FUNCTION-29}{\texttt{definition-property}}
    called \texttt{source-location}. If present, it must be a function
    of no arguments that returns a source location or \texttt{nil}.
    Typically, this is set up in the defining macro like this:

\begin{Shaded}
\begin{Highlighting}[]
\NormalTok{(}\KeywordTok{setf}\NormalTok{ (definition{-}property xref }\DataTypeTok{\textquotesingle{}source{-}location}\NormalTok{)}
\NormalTok{      (this{-}source{-}location))}
\end{Highlighting}
\end{Shaded}
  \end{itemize}

  This function is for extension only. Do not call it directly.
\end{itemize}

\subsubsection{Definition Properties}\label{definition-properties}

\label{x-28DREF-EXT-3A-40DEFINITION-PROPERTIES-20MGL-PAX-3ASECTION-29}

Arbitrary data may be associated with definitions. This mechanism is
used by
\paxlink{x-28DREF-EXT-3AARGLIST-2A-20GENERIC-FUNCTION-29}{\texttt{arglist*}},
\paxlink{x-28DREF-EXT-3ADOCSTRING-2A-20GENERIC-FUNCTION-29}{\texttt{docstring*}}
and
\paxlink{x-28DREF-EXT-3ASOURCE-LOCATION-2A-20GENERIC-FUNCTION-29}{\texttt{source-location*}}
for easy extension.

The following functions take an \texttt{xref} argument and not a
\texttt{dref} to allow working with non-canonical or non-existent
definitions.

\begin{itemize}
\item
  \paxlocativetypewithsource{https://github.com/melisgl/mgl-pax/blob/master/dref/src/base/extension-api.lisp\#L1200}{function}\paxname{definition-property}\phantomsection\label{x-28DREF-EXT-3ADEFINITION-PROPERTY-20FUNCTION-29}
  \emph{xref indicator}

  Return the value of the property associated with \texttt{xref} whose
  name is
  \texttt{eql}(\href{http://www.lispworks.com/documentation/HyperSpec/Body/f_eql.htm}{\texttt{0}}
  \href{http://www.lispworks.com/documentation/HyperSpec/Body/t_eql.htm}{\texttt{1}})
  to \texttt{indicator}. The second return value indicates whether the
  property was found.
  \href{http://www.lispworks.com/documentation/HyperSpec/Body/m_setf_.htm}{\texttt{setf}}able.
\item
  \paxlocativetypewithsource{https://github.com/melisgl/mgl-pax/blob/master/dref/src/base/extension-api.lisp\#L1220}{function}\paxname{delete-definition-property}\phantomsection\label{x-28DREF-EXT-3ADELETE-DEFINITION-PROPERTY-20FUNCTION-29}
  \emph{xref indicator}

  Delete the property associated with \texttt{xref} whose name is
  \texttt{eql}(\href{http://www.lispworks.com/documentation/HyperSpec/Body/f_eql.htm}{\texttt{0}}
  \href{http://www.lispworks.com/documentation/HyperSpec/Body/t_eql.htm}{\texttt{1}})
  to \texttt{indicator}. Return true if the property was found.
\item
  \paxlocativetypewithsource{https://github.com/melisgl/mgl-pax/blob/master/dref/src/base/extension-api.lisp\#L1243}{function}\paxname{definition-properties}\phantomsection\label{x-28DREF-EXT-3ADEFINITION-PROPERTIES-20FUNCTION-29}
  \emph{xref}

  Return the properties of \texttt{xref} as an association list.
\item
  \paxlocativetypewithsource{https://github.com/melisgl/mgl-pax/blob/master/dref/src/base/extension-api.lisp\#L1239}{function}\paxname{delete-definition-properties}\phantomsection\label{x-28DREF-EXT-3ADELETE-DEFINITION-PROPERTIES-20FUNCTION-29}
  \emph{xref}

  Delete all properties associated with \texttt{xref}.
\item
  \paxlocativetypewithsource{https://github.com/melisgl/mgl-pax/blob/master/dref/src/base/extension-api.lisp\#L1230}{function}\paxname{move-definition-properties}\phantomsection\label{x-28DREF-EXT-3AMOVE-DEFINITION-PROPERTIES-20FUNCTION-29}
  \emph{from-xref to-xref}

  Associate all properties of \texttt{from-xref} with \texttt{to-xref},
  as if readding them one-by-one with
  \texttt{(setf\ definition-property)}, and deleting them from
  \texttt{from-xref} with
  \paxlink{x-28DREF-EXT-3ADELETE-DEFINITION-PROPERTY-20FUNCTION-29}{\texttt{delete-definition-property}}.
\end{itemize}
